
    The problem is to optimize and parallelize a particle simulator, we
    where given four implementations of the simulator: one running sequentially,
    one parallelized using pthreads, one parallelized using openmp and one
    parallelized using MPI. The given simulator implementation made $n^2$
    computations each frame, where $n$ is the number of particles.

    But in contrast to the traditional N-body problem this simulation only
    simulates the interaction between particles within a certain distance from
    each other, moreover the density of particles are set low enough so that
    only $O(n)$ interactions are expected at any given moment. Therefore it
    should be possible to do the necessary calculations for each frame in
    $O(n)$, this is the first task. Once all implementation runs the
    computations in $O(n)$ %TODO: ... Time to optimize the parallel implementations.

    % Not needed?
    %\subsection{Not quite N-body problem} %TODO: Dig in deeper on the math, lead into the solution explained in the next section. How we can make this O(n).
